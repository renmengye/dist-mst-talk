% !TEX root=../presentation_1.tex
\section{Method}

\begin{frame}
\frametitle{A Simple Deterministic Algorithm [Elk17]}
\begin{itemize}
    \item Now, we can look at this paper.
    \item The idea of this paper is that we can use better BFS broadcasting structure.
    \item Suppose we have already computed an $(n/k,O(k))$-MST forest $\mathcal{F}$ using GHS. 
    \item This means we have at most $n/k$ fragments, each with diameter $O(k)$.
    \item This step takes $O(k \log^*n)$ rounds with $O(m + k \log n)$ messages.
\end{itemize}
\end{frame}


\begin{frame}
\frametitle{A Simple Deterministic Algorithm [Elk17]}
\begin{itemize}
    \item First we build an auxillary BFS tree $\tau$.
    \item This step takes $O(D)$ time and $O(m)$ messages.
    \item Then for each node, we compute the interval of the node based on the BFS tree.
    \begin{itemize}
        \item One convergecast to count the number of leaves under each node.
        \item One broadcast to assign interval.
        \item Takes $O(D)$ time and $O(n)$ messages.
    \end{itemize}
    \item Interval is computed so that we know how to route messages from the root to any base fragment F: $v$ finds a child $u$ such that $I(u)$ covers $I(F)$.
    \item Finally conduct a pipelined convergecast and the root learns the $\frac{n}{k}$ intervals of all the base fragments. Sends messages of the maximum interval of each fragment.
    \item Takes $O(D+\frac{n}{k})$ and $O(D \cdot \frac{n}{k})$ messages. (Need to aggregate messages of one fragment at a time.)
\end{itemize}
\end{frame}

\begin{frame}
\frametitle{A Simple Deterministic Algorithm [Elk17]}
\begin{itemize}
    \item To send the fragment identity, it takes $O(1)$ time and $O(m)$ messages.
    \item To convergecast the $\frac{n}{k}$ fragment identities, it takes $O(D+\frac{n}{k})$ time and $O(D \cdot\frac{n}{k})$ messages.
    \item We can now assume that each vertex $v$ knows the identity of the base fragment it belongs to, and also the base fragments of its neighbors.
    \item Now suppose $D \le \sqrt{n}$, and we select $k=\sqrt{n}$.
    \item Since we know the fragment identity of the neighbors, we can compute MWOE locally, and sends the information to the root of the fragment. This takes $O(k)=O(\sqrt{n})$ rounds and $O(n)$ messages.
\end{itemize}
\end{frame}


\begin{frame}
\frametitle{A Simple Deterministic Algorithm [Elk17]}
\begin{itemize}
    \item After each fragment knows its own MWOE, we can perform a pipelined convergecast to the root. Each time a node forward to its parent the lightest edge for each fragment, if the fragment is routed through this node (remember we computed the interval).
    \item This takes $O(D+\frac{n}{k})$ rounds and $O(D \cdot \frac{n}{k})$ messages.
    \item Then the root locally computes the MWOE for every fragment, and tells each fragment which one to merge with.
    \item Sends message $<F,F'>$ through pipelined broadcast, to tell $F$ to merge with $F'$.
    \item The root node that receives merge message, changes the fragment identity.
    \item The pipelined broadcast takes $O(D+\frac{n}{k})$ rounds and $O(D \cdot \frac{n}{k})$ messages.
\end{itemize}
\end{frame}


\begin{frame}
\frametitle{A Simple Deterministic Algorithm [Elk17]}
\begin{itemize}
    \item Finally, every vertex need to notify its neighbor about the change in fragment identity. This takes $O(1)$ time and $O(m)$ messages.
    \item In total, the overall time complexity is $O(D+\frac{n}{k}) + O(k\log^* n) + O((D + k + \frac{n}{k})\log n) = O(\sqrt{n}\log n)$.
    \item The overall the message complexity is $O(E \log n + n\log n \cdot \log^* n)$.
\end{itemize}
\end{frame}

\begin{frame}
\frametitle{A Simple Deterministic Algorithm [Elk17]}
\begin{itemize}
    \item If $D > \sqrt{n}$, then let $k=D$.
    % \item Base fragments can be computed in $O(D \log^* n)$ rounds and $O(m + n\log n)$ messages.
    \item Then we use the same algorithm.
    \item In total, it requires $O(D+\frac{n}{k}) + O(k\log^* n) + O((D + k + \frac{n}{k})\log n) = O(D\log n)$ time.
    \item The number of messages is the same as before.
    \item Combining two cases, we have $O((D + \sqrt{n})\log n)$ time and $O(E \log n + n\log n \cdot \log^* n)$ messages.
\end{itemize}
\end{frame}


\begin{frame}
\frametitle{Conclusion}
\begin{itemize}
    \item Near optimal time and message complexities (optimal within a polylog factor).
    \item A fully deterministic algorithm.
    % \item Randomized algorithms are known to achieve optimal [PRS16].
\end{itemize}

% !TEX root=../presentation_1.tex
\begin{table}
\renewcommand{\arraystretch}{1.3}
\caption{Summary of the complexity of distributed MST algorithms}
\begin{tabular}{|c|c|c|}
\hline
Method         & Time Complexity            & Message Complexity\\
\hline
\hline
Pipeline-MST   & $O(D + n)$                 & $O(m + n^2)$\\
\hline
GHS            & $O(n \log n)$              & $O(m + n \log n)$\\
\hline
Controlled-GHS & $O(D + \sqrt{n} \log^* n)$ & $O(m + n^{\frac{3}{2}})$\\
\hline
\textbf{This Work}& $O((D + \sqrt{n}) \log n)$  & $O(m \log n + n \log n \log^* n)$ \\
\hline
\hline
Randomized [PRS16]& $\tilde{O}(D + \sqrt{n})$  & $\tilde{O}(m)$ \\
\hline
Lower Bound [PRS16] & $\tilde{\Omega}(D + \sqrt{n})$  & $\Omega(m)$ \\
\hline
\end{tabular}
\end{table}

\end{frame}
